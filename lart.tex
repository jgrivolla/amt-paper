%  Linguistics article environments in LaTeX
%  R.L.H. - EUROTRA PROJECT, Dept. Lang & Ling, University of Essex
%  Mar 16 90
%	Revised Aug 90
%
%
%  Input this file into your source file by putting the LaTeX command
%  `` \input <thisfilename> '' on its first line.
%
% `%' starts a comment !

%%	*********** Macro Definitions **************
%%
% LINGUISTIC ITEM ENVIRONMENT
% Linguistic items (examples, diagrams, trees etc) are normally
% numbered consecutively throughout a linguistic article.

\newcounter{licntr}			% linguistic item counter

% Two commands required in the defintion of a linguistic item
\newcommand{\stepli}{\refstepcounter{licntr}}	% increment
					  	% li counter
\newcommand{\lival}{\thelicntr}			% return value
						% of li
						% counter
%% Now we define the ``linguistic item'' environment for cases where
%% there is just one \item to be displayed e.g. a single sentence
%% First, define the outer environment (which handles the numbering)
\newenvironment{liout}{\small
		\stepli			% step li counter
		\samepage			% prevent page break
		\begin{list}{		% begin definition
			(\lival)\hfill}{} 	% left adjust number
			\samepage
			\item 			% dummy item
		}{ \end{list}}

\newenvironment{liexc}{\small
		\samepage			% prevent page break
		\begin{list}{		% begin definition
			(27)\hfill}{} 	% left adjust number
			\samepage
			\item 			% dummy item
		}{ \end{list}}

%% Now embed a new list in the outer list - to typographical match the
%% ``lis'' environment (where we have alpha-indexed items)
\newenvironment{li}{\small
		\begin{liout} 			% begin outer list
		\begin{list}{\samepage \noindent}{} 	% begin inner
						% list,  prevent page break 
		\item}{				% item
		\end{list} 
		\end{liout}}

\newenvironment{liexcp}{\small
		\begin{liexc} 			% begin outer list
		\begin{list}{\samepage \noindent}{} 	% begin inner
						% list,  prevent page break 
		\item}{				% item
		\end{list} 
		\end{liexc}}



%% Special version of li which does NOT indent material - puts number
%% on line by itself and then (unindented) material below
\newenvironment{li*}{\small
	\stepli
	\samepage				% prevent page break
	\begin{trivlist} \item[] (\lival) \end{trivlist}
	\begin{trivlist} 
	{\samepage \item[]}
	}{\end{trivlist}}					% space
				

% LINGUISTIC ITEMS ENVIRONMENT
% Define a counter for alphabetically indexed items
\newcounter{alphcntr}		% will be reset in each new lis environment

% Define an environment (called ``lis'' = linguistic items) for a
% series of alpha-indexed linguistic items e.g. 
%	(1) a.
%	    b.    etc.
\newenvironment{lis}{\small
		\begin{liout} 				% begin outer list
			\begin{list}{			% begin inner
							% list
				\samepage		% prevent pg breaks
				\alph{alphcntr}.\hfill 	%left-align letter
					}{
				\usecounter{alphcntr}}}{
		\end{list} \end{liout}}

% LINGUISTIC PRINCIPLE COMMAND - \princ
\newtheorem{prin}{Principle}
%%\newtheorem{ax}{Axiom}
%%\newtheorem{fig}{Figure}
\newtheorem{assum}{Assumption}
\newtheorem{defn}{Definition}
% Define any more you want here.....

% Now define a command with two arguments to handle linguistic
% principles.  The command - ``princ'' takes two arguments: the
% first is the name associated with a \newtheorem (e.g. ``prin'') and
% the second  is the text of that principle. (We use a command here
% rather than an  environment due to technical limitations of LaTeX environment
% parameter handling.)
\newcommand{\princ}[2]{\begin{liout} 
			\begin{#1} 
				{#2}
			\end{#1}
			\end{liout}}
			
			
% MACROS FOR PRODUCING LABELLED/UNLABELLED BRACKETS
\newcommand{\lb}{\mbox{$ [ $}}			% unlabelled right
						% bracket 

\newcommand{\rb}{\mbox{$ ] $}}			% unlabelled right
						% bracket 

\newcommand{\llb}[1]{\mbox{$ [_{#1} $}}		% labelled left
						% bracket

\newcommand{\lrb}[1]{\mbox{$ ]_{#1} $}}   	% labelled right
					      	% bracket

% MACROS FOR HANDLING LISTS, SHORTCUTS etc - A.W., 13/1/95
\newcommand{\bi}{\begin{itemize}}		 
\newcommand{\ei}{\end{itemize}}		 

\newcommand{\be}{\begin{enumerate}}		 
\newcommand{\ee}{\end{enumerate}}

\newcommand{\bc}{\begin{center}}		
\newcommand{\ec}{\end{center}}		

\newcommand{ \heading }[1]{ \begin{center}
			   {\bf {#1}}
			    \end{center}}

% MACROS FOR STANDARD LINGUISTIC SYMBOLS (can be used in text mode)
\newcommand{\tarr}{\mbox{$ \Longrightarrow $}}	% ==> i.e
						% translation arrow

\newcommand{\rarr}{\mbox{$ \longrightarrow $}}	% --> i.e.
						% rule arrow

\newcommand{\llr}{\mbox{$ \longleftrightarrow $}}	% <--> i.e.
						% rule arrow

\newcommand{\up}{\mbox{$ \uparrow $}}	% uparrow for text/maths
						% environment

\newcommand{\down}{\mbox{$ \downarrow $}}	% downarrow for
					    	% text/maths environment

%% \newcommand{\th}{\mbox{$ \theta $}}		% theta
\newcommand{\al}{\mbox{$ \alpha $}}		% alpha

% MACROS FOR ASTERISKING LINGUISTIC DATA
% (ensures data strings are neatly aligned e.g
%	Green ideas sleep furiously
%      *Green ideas furious sleep

\newlength{\astspace}				% initialise length command

\settowidth{\astspace}{\mbox{$\ast$}}		% set length command
						% to width of an asterisk

\newcommand{\un}{\mbox{$\ast$}}			% unacceptable string (*)

\newcommand{\qu}{?}				% questionable string

\newcommand{\ac}{\hspace*{\astspace}}		% acceptable string
						% (inserts *-width
						% space at start of
						% string)

%% ATTRIBUTE-VALUE ARRAY ENVIRONMENT		
%% Now we define some specialised linguistic items, starting with an
%% attribute value array
\newcommand{\attrib}[1]{{\rm \bf {#1}}}		% define the
						% attributes of an
						% attribute

\newcommand{\atomval}[1]{{\rm {#1}}}		% define attribute of
						% an atomic value

\newcommand{\sematomval}[1]{{\mit {#1}}}	% define attribute of 
						% semantic atomic values

% Att-Val array environment.  
% Must occur inside maths environment
\newenvironment{av}{\left[ \begin{array}{ll}}{\end{array} \right]}

%% Labelled Att-Val array environment
\newenvironment{lav}[1]{ \left[_{#1} \begin{array}{ll}}{\end{array} \right]}


%% TREE ENVIRONMENT - \begin{tree}{}......\end{tree}
% Trees produced by specialised tabular environment which takes an
% argument corresponding to the width of the tree - a string in c*
% e.g. cccc  for a tree 4 nodes wide. Lines between nodes must be   
% drawn by hand. 
\newenvironment{tree}[1]{		
	\begin{large}				% nodes in large typeface
	\renewcommand{\arraystretch}{4}		% set interrow space
						% to 4 (LaTeX default
						% is 1.  Change this
						% value to increase
						% depth of subtrees.
						% Range 3-6  approx

	\begin{tabular}{{#1}} 			% begin tree with 
						% argument-specified width
	\addtolength{\tabcolsep}{\tabcolsep}	% Double normal table
					    	% column separation. 
	}{				

	\addtolength{\tabcolsep}{-0.5\tabcolsep}	% Restore normal table
					     	% column separation
	\end{tabular}				% end of tree description

	\renewcommand{\arraystretch}{1}		% Restore default
						% array row separation
	\end{large} }				% Back to normal size

%% Commands for drawing boxes around lists, from Arnold et al '94.

\newcommand{\dougswideboxfigure}[3]{	
\begin{figure}[hbtp]
\begin{center}
\begin{tabular}{|p{134mm}|}\hline
%{|@{\hspace{3em}}l@{\hspace{3em}}|}\hline
\\
\multicolumn{1}{|c|}{\bf #2}\\
% only one extra line here is better:\\
\begin{center}
\begin{minipage}[t]{110mm}
#3
\end{minipage}
\end{center}
\\
\hline
\end{tabular}
\end{center}
\label{#1}
\end{figure}
}



\newcounter{figcntr}			% figiom counter

% Two commands required in the defintion of an figiom
\newcommand{\stepfig}{\refstepcounter{figcntr}}	% increment
					  	% fig counter
\newcommand{\figval}{\thefigcntr}		% return value
						% of fig
						% counter

%% Now we define the ``linguistic item'' environment for cases where
%% there is just one \item to be displayed e.g. a single sentence
%% First, define the outer environment (which handles the numbering)
\newenvironment{figout}{
		\stepfig			% step fig counter
		\begin{list}{		% begin definition
			(\figval)\hfill}{} 	% left adjust number
			\item 			% dummy item
		}{ \end{list}}

%% Now embed a new list in the outer list - to typographical match the
%% ``lis'' environment (where we have alpha-indexed items)
\newenvironment{fig}{
		\samepage			% prevent page break
		\begin{figout} 			% begin outer list
		\begin{list}{}{} 		% begin inner list
		\item}{				% item
		\end{list} 
		\end{figout}}

\renewcommand{\thefigcntr}{\Alph{figcntr}}

\newcounter{axcntr}			% axiom counter

% Two commands required in the defintion of an axiom
\newcommand{\stepax}{\refstepcounter{axcntr}}	% increment
					  	% ax counter
\newcommand{\axval}{\theaxcntr}		% return value
						% of ax
						% counter

%% Now we define the ``linguistic item'' environment for cases where
%% there is just one \item to be displayed e.g. a single sentence
%% First, define the outer environment (which handles the numbering)
\newenvironment{axout}{
		\stepax			% step ax counter
		\begin{list}{		% begin definition
			(\axval)\hfill}{} 	% left adjust number
			\item 			% dummy item
		}{ \end{list}}

%% Now embed a new list in the outer list - to typographical match the
%% ``lis'' environment (where we have alpha-indexed items)
\newenvironment{ax}{
		\samepage			% prevent page break
		\begin{axout} 			% begin outer list
		\begin{list}{\noindent}{} 		% begin inner list
		\item}{				% item
		\end{list} 
		\end{axout}}

\renewcommand{\theaxcntr}{\Roman{axcntr}}

%EMBEDDED AXIOMS e.g.
%	(1) a.
%	    b.    etc.

\newcounter{alphacntr}		% will be reset in each new ax environment

\newenvironment{axs}{
		\begin{axout} 				% begin outer list
			\begin{list}{			% begin inner
							% list
				\samepage		% prevent pg breaks
				\alph{alphacntr}.\hfill %left-align letter
					}{
				\usecounter{alphacntr}}}{
		\end{list} \end{axout}}


\newcommand{\s}{\mbox{$ \sectionsymbol $}}      
\let\sectionsymbol=\S
\def\S{{}\;{}}
